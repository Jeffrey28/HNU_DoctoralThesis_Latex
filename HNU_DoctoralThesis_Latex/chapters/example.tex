\chapter{论文模板简单使用说明}
\section{Latex使用}
该模板已经在mac的texifier,overleaf,vscode上验证使用,请用XeLaTex进行编译。

\section{参考文献}

在Doctoral-thesis.tex修改参考文献链接:\lstinline!\bibliography{MyLibrary.bib}!。

目前可以使用的引用命令有\lstinline!\cite{}!、\lstinline!\citep{}!、\lstinline!\citet{}!和\lstinline!\citen{}!

效果如下:

引用 ResNet~\cite{2020_aradi_Survey}提出了ResNet网络结构\citep{2020_aradi_Survey}(上述第一个效果和第二个效果相同)。

以人名为主语:\citet{2017_qinxiaohui_FeiYunZhiCheLiangDuiLieDeFenBuShiKongZhi}针对队列稳定性提出了观测性方法。或者\citet{2021_chen_Graph}对与图做了个总结。

参考文献的引用图标不上标:\citen{2017_qinxiaohui_FeiYunZhiCheLiangDuiLieDeFenBuShiKongZhi}这样。

\section{图表}
插入图的常用命令有:
\begin{lstlisting}
\begin{figure}[htb]
  \centering
  \includegraphics[width=.7\textwidth, page=1]{figures/dc.pdf}
  \caption{决策控制框架}
  \label{fig:dc-1}
\end{figure}
\end{lstlisting}

引用的使用可以直接使用\lstinline!\figref{fig:dc-1}!

三线表的命令:
\begin{lstlisting}
\begin{table}[htb]
  \centering
  \caption{训练超参数}
  \label{tab:sarl-4}
  \begin{tabular}{ccc}
	\toprule
	物理意义 & 英文/符号 & 数值 \\ 
	\midrule
	 隐含层数 & Hidden layer number & $\rm 2$ \\
	\bottomrule
  \end{tabular}
\end{table}
\end{lstlisting}

引用格式为\lstinline!\tabref{tab:sarl-4}!

\section{定理类环境}
该模板定义了定理,引理,评注等,具体有:
\begin{lstlisting}
\newcommand\hnu@assertionname{断言}
\newcommand\hnu@axiomname{公理}
\newcommand\hnu@corollaryname{推论}
\newcommand\hnu@definitionname{定义}
\newcommand\hnu@propertyname{性质}
\newcommand\hnu@examplename{例}
\newcommand\hnu@lemmaname{引理}
\newcommand\hnu@proofname{证明}
\newcommand\hnu@propositionname{命题}
\newcommand\hnu@remarkname{评注}
\newcommand\hnu@theoremname{定理}
\newcommand\hnu@assumptionname{假设}
\end{lstlisting}

\section{公式引用}
引用格式为\lstinline!\eqref{}!

\begin{figure}
    \centering
    \includegraphics{HNU_DoctoralThesis_Latex/figures/hnu-logo.png}
    \caption{Caption}
    \label{fig:1}
\end{figure}

\begin{figure}
    \centering
    \includegraphics{HNU_DoctoralThesis_Latex/figures/hnu-logo.png}
    \caption{Caption}
    \label{fig:1}
\end{figure}

\begin{algorithm} [htb]                                            
	\caption{FCFS算法}
  	\label{algo:revi-1}  
  	\SetKwInOut{KwIn}{初始化}
    \SetKwInOut{KwOut}{输出}
	\KwIn{交叉路口信息,即网格占用点}
	得到交叉路口车辆进入序列 \\
	\For{\rm $1 < j < N$}{
		\For{\rm $k<j$}{
			识别车辆$j$与$k$之间的公共占用点$c_{j,k}\in \{1,2,\dots,Q_j\}$ \\
			获得车辆$k$的进入时间$t_k^{\rm in}$ \\
			$t_{j, k}^{\rm in}=t_k^Q+\frac{S}{V_{\text {int}}}+\delta$ \quad $\rhd$ 假设$Q$是占用点
			}
		$t_j^{\mathrm{in}}=\max \left\{t_{j, 1}^{\mathrm{in}}, t_{j, 2}^{\mathrm{in}}, \ldots, t_{j, k}^{\mathrm{in}}\right\}$ \quad $\rhd$ 获得最保守的进入时间 \\
		随后更新并存储$t_j^{\mathrm{in}}, t_j^1, t_j^2, \ldots t_j^{Q_j}, t_j^{\text {out }}$。
		}
\end{algorithm}


\begin{table}[htb]
  \small
  \centering
  \caption{单车强化学习MDP设置}
  \label{tab:revi-1}
  \begin{tabular}{cc}
	\toprule[1.5pt]
	名称 & 参数  \\ 
	\midrule[0.75]
	状态 & \makecell{鸟瞰图 \\ 显示航向角和速度的图像快照 \\ 车辆特征向量和信号特定信息} \\
	动作 &  \makecell{加速度 \\ 等待、慢慢前进、走 \\ 向上、向下、向左和向右移动} \\
	奖励 & \makecell{成功到达奖励和碰撞惩罚 \\ 速度、跟车间距和预计到达交叉路口时间的函数} \\
	\bottomrule[1.5pt]
  \end{tabular}
\end{table}
